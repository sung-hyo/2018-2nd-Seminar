%\documentclass[10pt]{beamer}
%\mode<presentation>

%\usetheme{Warsaw}
\documentclass[10pt]{beamer}

\usetheme[progressbar=frametitle]{metropolis}
\usepackage{appendixnumberbeamer}

\usepackage{kotex}
\usepackage{multicol}
\usepackage{url}
\usepackage{graphicx}
\usepackage{hyperref}

\title{\textbf{Data with storytelling}}
\subtitle{데이터의 시각화(Data Visualization)}
\author{SungHyo Seo}
\institute{Department of Information and Statistics}

\begin{document}
\maketitle

	\begin{frame}{Table of contents}
 		\tableofcontents
	\end{frame}

	\section{The What, Why}
	\begin{frame}
	\frametitle{What is Storytelling and Data Visualization}
		\begin{block}{스토리텔링(Storytelling)}
			\begin{itemize}
				\item 단어, 이미지, 소리를 통해 이야기를 전달하는 것
			\end{itemize}
		\end{block}
		\begin{block}{데이터 시각화(Data Visualization)}
			\begin{itemize}
				\item 데이터 분석 결과를 쉽게 이해할 수 있도록 시각적으로 표현하고 전달되는 과정
				\item 수많은 정보들을 시각적으로 묘사하고 필요한 정보를 효율적이고 명확하게 제공
			\end{itemize}
		\end{block}
	\end{frame}

	\begin{frame}
	\frametitle{데이터 스토리텔링이 필요한 이유}
		\begin{figure}
			\includegraphics[height=0.25\textwidth]{data_storytellers.png}
		\end{figure}
	\begin{itemize}
		\item 데이터는 강력한 힘을 갖지만, 이를 이야기로 잘 전달했을 때만이 오래도록 기억에 남을 수 있다.
		\item 데이터와 이야기가 함께 사용될 때 논리적이면서도 감성적으로 잠재고객에게 다가갈 수 있다
		\item 데이터와 분석은 그 자체로 충분하지 않으며, 사람들이 분석을 제대로 이해할 수 있도록 하는 것이 중요하다.
		\item 커뮤니케이션 없는 데이터는 무용지물이다
		\item 데이터 과학자의 진짜 역할은 스토리텔링이다.
	\end{itemize}
	\end{frame}

\begin{frame}
\frametitle{스토리텔링하는데 시각화가 필요한 이유}
\begin{itemize}

	\item 우리의 뇌는 일상 및 업무에서 생성되거나 처리되는 방대한 양의 데이터를 모두 처리하고 이해할 수 없다.
	\item 데이터에 숨겨진 모든 정보를 이해하지 못한 채 다양한 의사 결정을 내리게 된다.\\
	\item 데이터를 이용한 시작적 분석은 우리의 두뇌가 기존에 해결 할 수 없었던 대량의 정보를 처리하고, 수용하고, 해석할 수 있게 해준다.

\end{itemize}
	\begin{figure}
		\includegraphics[height=0.25\textwidth]{visual_0.png}
	\end{figure}
\end{frame}

\begin{frame}
\frametitle{데이터 시각화}
\begin{itemize}
	\item 통계분석기법으로 알 수 없는 데이터의 이야기를 이끌어내어 새로운 인사이트를 도출
	\item 의사결정자에 맞게 시각적 목적을 설정하고 데이터의 표현에 집중
	\item 데이터 시각화 후 스토리텔링 과정을 통해 완성
\end{itemize}
	\begin{figure}
		\includegraphics[height=0.4\textwidth]{story_process.png}
	\end{figure}

\end{frame}

\begin{frame}
\frametitle{데이터 시각화 종류}
\begin{itemize}
	\item 크게 구분하여 시각화는 정보형과 서술형으로 구분
\end{itemize}
	\begin{figure}
		\includegraphics[height=0.4\textwidth]{visualization_1.png}
	\end{figure}

\end{frame}

\section{Data Visualization Example}
\begin{frame}
\frametitle{Information Graphics}
	\begin{itemize}
		\item 인포메이션 그래픽(인포그래픽, Information Graphics)
	\end{itemize}
	\begin{figure}
		\includegraphics[height=0.6\textwidth]{visual_1.jpg}
		\includegraphics[height=0.6\textwidth]{infographic2.jpg}
	\end{figure}
\end{frame}

\begin{frame}
\frametitle{나폴레옹 진군 지도}
\begin{itemize}
	\item 선의 너비가 군인의 총 수
	\item 색상은 방향(노란색: 모스크바로의 출정, 검은색: 복귀)
	\item 중앙 아래에는 간단한 온도 라인 그래프
\end{itemize}
	\begin{figure}
		\includegraphics[height=0.4\textwidth]{napolean_minard.png}
	\end{figure}
그 여정이 얼마나 끔찍한 참사였는지를 충격적인 그림으로 그려내고 있습니다.
\end{frame}

\begin{frame}
\frametitle{1854년 Broad 가의 콜레라 발병 맵}
\begin{itemize}
	\item 콜레라로 사망한 인원 수를 도시 구역에 막대 그래프로 사용
	\item 막대의 집중도와 길이가 어떤 특정 도시 구역의 집단을 표시
	\item 그결과로 콜레라의 피해를 가장 많이 입은 세대들이 같은 우물을 식수로 사용하고 있다는 것을 발견
\end{itemize}
	\begin{figure}
		\includegraphics[height=0.5\textwidth]{snow-cholera-map.jpg}
	\end{figure}
\end{frame}

\begin{frame}
\frametitle{크림 정쟁 당시 사망원인}
\begin{itemize}
	\item 1850년대에 크림 전쟁 중에 군인 사망률이 높고 계속 증가
	\item 대부분의 사망이 열악한 병원 상태에서 기인한다는 것을 밝힘
	\item 나선형 차트의 어두운 부분이 총 사망자 수이고, 더 어두운 부분이 전투에서 사망한 경우
	\item 부식한 의료 조건이 많은 죽음에 원인이며 예방 가능하다는 인사이트를 얻음
\end{itemize}
	\begin{figure}
		\includegraphics[height=0.4\textwidth]{nightingale-mortality.jpg}
	\end{figure}
\end{frame}

\section{Dashboard}
\begin{frame}
\frametitle{Dashboard}
	\begin{block}{Dashboard}
	\begin{itemize}
	\item 자동차 계기판
	\item 다양한 데이터를 동시에 비교하고 모니터링할 수 있도록 한 공간에 표시된 여러 워크시트와 지원 정보의 모음
	\item 다른 이름으로 진행보고서(progress report), report
	\end{itemize}
	\end{block}
	
	\begin{block}{Dashboard 필요}
	\begin{itemize}
	\item 제공해야할 정보를 반복적으로 전달할 필요가 있을 때
	\item 데이터가 있는 곳과 필요로 하는 곳이 다를 때
	\end{itemize}
	\end{block}
\end{frame}

\begin{frame}
\frametitle{Dashboard}
	\begin{block}{대쉬보드 고려사항}
	\begin{itemize}
	\item 무슨 목적으로 누구를 위해 만드는가
	\item 목적에 부합하는 지표는 무엇인가
	\item 시각화,상호작용, 데이터 업데이트, 접근성
	\end{itemize}
	\end{block}
\end{frame}

\begin{frame}
\frametitle{Dashboard}
		\begin{itemize}
			\item TABLEAU X 경기도감염병관리본부의 감염발생현황
		\end{itemize}
		\begin{figure}
			\includegraphics[height=0.5\textwidth]{visual_2.png}
		\end{figure}
\end{frame}

\begin{frame}
\frametitle{대화형 정부 예산}
\begin{itemize}
	\item 오바마 대통령 재임 중에 작성된 이 트리맵은 미국 정부의 예산을 시각적으로 분류
	\item 이 시각화의 특징은 전달 방식에 있음
	\item 납세자들에게 그들의 세금이 어디로 가는지 의사소통하는 대화형 시각화를 사용했다는 점
\end{itemize}
	\begin{figure}
		\includegraphics[height=0.4\textwidth]{2016_budget_visualization_safe.jpg}
	\end{figure}
\end{frame}

\begin{frame}
\frametitle{도시 셀카}
\begin{itemize}
	\item 데이터의 광범위한 뷰를 하나의 초국가적인 현상의 컨텍스트로 나타냄
	\item 세계 곳곳의 12만 장의 셀카 사진을 분석
	\item 도시별로 머리 기울기, 유행하는 포즈, 련령 및 성별 등
\end{itemize}
	\begin{figure}
		\includegraphics[height=0.4\textwidth]{selfiecity.jpg}
	\end{figure}
\end{frame}

\begin{frame}
\frametitle{다가오는 모든 일식}
	\begin{itemize}
		\item 일식의 경로와 2080년까지 모든 미래의 일식 경로를 보여주는 대화형 지구 비주얼리제이션
		\item 돌아가는 지구는 태양이 달에 의해 완전히 가려지는 개기 일식의 경로와, 개기 일식이 지구상에서 일어날 지점과 시점(시간은 밝거나 어두운 음영 및 마우스오버 텍스트로 표시됨)을 보여줌
	\end{itemize}
	\url{https://www.washingtonpost.com/graphics/national/eclipse/? utm_term=.fd1b287c6d6b}
\end{frame}

\begin{frame}
\frametitle{데이터스토리를 보여주는 사이트}
	\begin{itemize}
		\item 파키스탄의 무인정찰기 ‘드론’의 민간인 피해를 보여주는 ‘out of sight, out of mind \\
		\url{http://drones.pitchinteractive.com/}
		\item Information is Beautiful Awards \\
		\url{https://www.informationisbeautifulawards.com/}
	\end{itemize}
\end{frame}

\end{document}